\documentclass[11pt,a4paper]{moderncv}

% moderncv themes
%\moderncvtheme[blue]{casual}                 % optional argument are 'blue' (default), 'orange', 'red', 'green', 'grey' and 'roman' (for roman fonts, instead of sans serif fonts)
\moderncvtheme[blue]{classic}                % idem
\usepackage{xunicode, xltxtra}
\XeTeXlinebreaklocale "zh"
\widowpenalty=10000

%\setmainfont[Mapping=tex-text]{文泉驿正黑}

% character encoding
%\usepackage[utf8]{inputenc}                   % replace by the encoding you are using
\usepackage{CJKutf8}

% adjust the page margins
\usepackage[scale=0.8]{geometry}
\recomputelengths                             % required when changes are made to page layout lengths
\setmainfont[Mapping=tex-text]{Hiragino Sans GB}
\setsansfont[Mapping=tex-text]{Hiragino Sans GB}
\CJKtilde

% personal data

%% start of file `template-zh.tex'.
%% Copyright 2006-2012 Xavier Danaux (xdanaux@gmail.com).
%
% This work may be distributed and/or modified under the
% conditions of the LaTeX Project Public License version 1.3c,
% available at http://www.latex-project.org/lppl/.

% 个人信息
\firstname{李}
\familyname{肇中}
\title{个人简历}                      % 可选项、如不需要可删除本行
\address{中关村南四街4号}{100190 北京}             % 可选项、如不需要可删除本行
\mobile{+86~18510253518}                         % 可选项、如不需要可删除本行
%\phone{+2~(345)~678~901}                          % 可选项、如不需要可删除本行
%\fax{+3~(456)~789~012}                            % 可选项、如不需要可删除本行
\email{ldm5235@gmail.com}                    % 可选项、如不需要可删除本行
\homepage{lizhaozhong.info}                  % 可选项、如不需要可删除本行
%\extrainfo{附加信息 (可选项)}                  % 可选项、如不需要可删除本行
\photo[64pt]{avatar2.jpg}                  % ‘64pt’是图片必须压缩至的高度、‘0.4pt‘是图片边框的宽度 (如不需要可调节至0pt)、’picture‘ 是图片文件的名字;可选项、如不需要可删除本行
%\quote{引言(可选项)}                           % 可选项、如不需要可删除本行

% 显示索引号;仅用于在简历中使用了引言
%\makeatletter
%\renewcommand*{\bibliographyitemlabel}{\@biblabel{\arabic{enumiv}}}
%\makeatother

% 分类索引
%\usepackage{multibib}
%\newcites{book,misc}{{Books},{Others}}
%----------------------------------------------------------------------------------
%            内容
%----------------------------------------------------------------------------------
\begin{document}
\maketitle

\section{工作经历}
\cventry{2016 -- 至今}{软件工程师}{中国科学院软件研究所}{}{}{}

\section{教育背景}
\cventry{2013 -- 2016}{硕士}{兰州大学}{信息科学与工程学院}{计算机技术}{}
\cventry{2009 -- 2013}{本科}{电子科技大学}{信息与软件工程学院(示范性软件学院)}{软件工程(实验班)}{3.49/4}

%\section{毕业论文}
%\cvitem{题目}{\emph{题目}}
%\cvitem{导师}{导师}
%\cvitem{说明}{\small 论文简介}

\section{社区}
\cventry{Blog}{\url{http://www.lizhaozhong.info}}{技术博客}{}{}{}
\cventry{GitHub}{\url{http://github.com/lzz5235}}{}{}{}{}
\cventry{Git page}{\url{http://lzz5235.github.io/}}{}{}{}{}

\section{项目经历}
\renewcommand{\baselinestretch}{1.2}
\cventry{2016.8 -- 至今}
{某海军作战系统任务规划模块开发}
{C++,Qt}
{}{项目核心成员}
{该海军作战系统基于已有的内部类库开发,每个功能模块为一个lib文件。本人完成该任务规划模块的界面设计和数据库交互,按照需求设计数据库表结构,配合该作战系统的其他子模块完成既定需求。}

\cventry{2016.7 -- 2016.8}
{某海军后勤管理系统国产化移植和性能优化}
{Java,JavaFx,Oracle}
{}{}
{该作战系统基于 Java Swing 开发,移植系统初期系统运行缓慢。该项目代码行数10W+,通过断点跟踪,分析系统执行路径和性能瓶颈,重写系统中与XML操作相关模块和失效命令,对关键数据结构HashMap进行对象序列化和反序列化,加快启动过程。将该系统数据库从Oracle迁移到国产数据库,添加关键索引以加快数据库存取速率。}

\cventry{2014.8 -- 2015.9}
{内存多Bit错误修复内核模块开发}
{Memory Kernel MCE}
{}{项目核心成员}
{分析当前MCA内核子系统的工作原理,修改Machine Check Exception的工作流程,发信号通知内核线程终止运行,以防止更加严重的panic。对于用户层的进程,使用PALLOC+LXC+CRIU 进行错误抑制与错误隔离。配合使用在线内存检测与离线内存检测方式,达到服务器的平稳运行,而不会因为MCE相关错误而中断服务。编写大量桩模块,进程/内核线程配合 MCA 的处理。}

\cventry{2014.3-7}
{JOS}
{C/Assemble Qemu}
{个人项目}{\url{https://github.com/lzz5235/JOS-kernel}}
{JOS 是一个Toy Kernel,可以运行在qemu中。JOS涵盖了操作系统的五大部分Booting/Memory management/User-level environments/Preemptive multitasking/File system/shell。通过实现这五大部分的代码,可以使得JOS运行起来。}

\cventry{2013.11--2014.1}
{ChatOnline}
{Qt C++}
{个人项目}{\url{https://github.com/lzz5235/ChatOnline}}
{ChatOnline基于Qt开发,开发的初衷就是满足实验室内部人员沟通需求,因为Linux与Windows互通没有成熟的解决方案,ChatOnline为解决实验室在Linux/Windows下消息互通方面存在的缺陷。该软件采用C/S架构,目前仅限于实验室内部使用,效果良好。}

\cventry{2013.9--2014.1}
{生活小贴士}
{Java}
{个人项目}{微信公众号lifestyle\_day}
{生活小贴士是一个基于微信公众平台开发的微信公众号,后台使用Java+Tomcat对于用请求进行解析,然后返回用户需要的结果。本公众号在本人的测试号中实现高级菜单,可以方便进行各种点击操作。某些特定功能API依赖Baidu等第三方API,使用dom4j包对返回数据进行解析。然后发送给用户。}

\cventry{2012.1-12}
{分布式网盘系统}
{epoll C++ Linux}
{学校创新创业项目}{服务端开发}
{使用可复用epoll非阻塞+线程池框架,配合Shell脚本,针对客户端请求进行操作,包括将用户数据多地存储,并将文件返回给用户。该框架允许更多的用户对网盘进行读写,提高服务器运行效率。}

\section{语言技能}
\cvline{英语}{\textbf{CET-6},擅长读写,经常阅读英文文档、教程,并与Kernel Maintainer Email交流。}
\cvline{普通话}{母语}

\section{计算机技能}
\cvline{编程语言}{C/C++ 设计模式 > Java > Shell > Python} 
\cvline{内核机制}{MCA, Signal, Kthread,Physical Memory Management, Interruption, VFS, CFS,Kernel Driver Development}
\cvline{Toolchain}{gcc, gdb, cscope, ctags, make, strace, Kdump Toolchain, LXC, Docker}
\cvline{Client框架}{Qt,MFC}
\cvline{工具}{Linux(Fedora,CentOS,Debian), Vim, Git, SVN,\LaTeX, Markdown, Apache}
\cvline{数据库}{MySQL,SQLite,Oracle}
\cvline{网页模版}{WordPress,Jekyll,XOOPS}
\cvline{其他工具}{Dia, Visio, PowerDesigner}


\section{实习经验及奖励}
\cvline{2015.6}{网易MOOC课程-软件测试方法和技术实践证书}
\cvline{2015.3}{网易MOOC课程-Python程序设计优秀证书}
\cvline{2014.12}{工信部人才紧缺职业技能认证- 高级 Linux 内核开发工程师}
\cvline{2014--2015}{兰州大学研究生一等奖学金}
\cvline{2014.5}{新浪SAE中级开发者证书}
\cvline{2013--2014}{兰州大学研究生一等奖学金}
\cvline{2012--2013}{电子科技大学人民三等奖学金}
\cvline{2012.2}{新加坡义安理工文化交流活动毕业证书}
\cvline{2011--2012}{IBM大型机AIX高级管理员证书}
\cvline{2011.7-8}{大连东软实训结业证书}

\section{校园经历} % (fold)
\cventry{2014-2015}
{培训Git多分支开发模型使用}{}{}{}{任项目CM(Configure Manager),包括webgit搭建,git 数据定期维护,具体开发流程参考Kernel Development。}
\cventry{2013-2014}
{DSLab 实验室网站管理}{}{}{}{DSLab网站的维护,数据迁移,从XOOPS -> WordPress 迁移。}
\cventry{2013-2014}
{Lanzhou Google Development Group Volunteer}{}{}{}{负责lanzhou GDG 社区的线上报名网站维护,\url{http://lanzhougdg.sinaapp.com} 。}
\cventry{2013--2014}
{兰大开源社区技术主讲人}{}{}{}{在任期间负责兰大开源技术交流。}
\cventry{2015.6-2015.7}
{elinux.org中文翻译计划志愿者}{}{}{}{eLinux.org 是 Linux 基金会下属 Consumer Electronics 
Linux Forum 维护的一个 Embedded Linux Wiki。该 Wiki 全面系统地梳理了嵌入式 Linux 方方面面的知识。我负责翻译该wiki下的Security章节。\url{https://github.com/lzz5235/elinux}}

\closesection{}                   % needed to renewcommands
\renewcommand{\listitemsymbol}{-} % change the symbol for lists

% 来自BibTeX文件但不使用multibib包的出版物
%\renewcommand*{\bibliographyitemlabel}{\@biblabel{\arabic{enumiv}}}% BibTeX的数字标签
\nocite{*}
\bibliographystyle{plain}
\bibliography{publications}                    % 'publications' 是BibTeX文件的文件名

% 来自BibTeX文件并使用multibib包的出版物
%\section{出版物}
%\nocitebook{book1,book2}
%\bibliographystylebook{plain}
%\bibliographybook{publications}               % 'publications' 是BibTeX文件的文件名
%\nocitemisc{misc1,misc2,misc3}
%\bibliographystylemisc{plain}
%\bibliographymisc{publications}               % 'publications' 是BibTeX文件的文件名

\end{document}


%% 文件结尾 `template-zh.tex'.
