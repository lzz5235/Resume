%%%%%%%%%%%%%%%%%
% This is Yoann E. A. Le Bihan's CV based on the altacv.cls template
% (v1.1.3, 30 April 2017) written by LianTze Lim (liantze@gmail.com).
% Now compiles with pdfLaTeX, XeLaTeX and LuaLaTeX.
% 
%% It may be distributed and/or modified under the
%% conditions of the LaTeX Project Public License, either version 1.3
%% of this license or (at your option) any later version.
%% The latest version of this license is in
%%    http://www.latex-project.org/lppl.txt
%% and version 1.3 or later is part of all distributions of LaTeX
%% version 2003/12/01 or later.
%%%%%%%%%%%%%%%%

%% If you need to pass whatever options to xcolor
\PassOptionsToPackage{dvipsnames}{xcolor}

%% If you are using \orcid or academicons
%% icons, make sure you have the academicons 
%% option here, and compile with XeLaTeX
%% or LuaLaTeX.
% \documentclass[10pt,a4paper,academicons]{altacv}

%% Use the "normalphoto" option if you want a normal photo instead of cropped to a circle
% \documentclass[10pt,a4paper,normalphoto]{altacv}

\documentclass[10pt,a4paper]{altacv}

%% AltaCV uses the fontawesome and academicon fonts
%% and packages. 
%% See texdoc.net/pkg/fontawecome and http://texdoc.net/pkg/academicons for full list of symbols.
%% 
%% Compile with LuaLaTeX for best results. If you
%% want to use XeLaTeX, you may need to install
%% Academicons.ttf in your operating system's font 
%% folder.


% Change the page layout if you need to
\geometry{left=1cm,right=9cm,marginparwidth=6.8cm,marginparsep=1.2cm,top=1.25cm,bottom=1.25cm,footskip=2\baselineskip}

% Change the font if you want to.

% If using pdflatex:
\usepackage{CJKutf8} 

% If using xelatex or lualatex:
\setmainfont{SimSun}

% Change the colours if you want to
\definecolor{Mulberry}{HTML}{72243D}
\definecolor{SlateGrey}{HTML}{2E2E2E}
\definecolor{LightGrey}{HTML}{666666}
\colorlet{heading}{Sepia}
\colorlet{accent}{Mulberry}
\colorlet{emphasis}{SlateGrey}
\colorlet{body}{LightGrey}

% Change the bullets for itemize and rating marker
% for \cvskill if you want to
\renewcommand{\itemmarker}{{\small\textbullet}}
\renewcommand{\ratingmarker}{\faCircle}

%% sample.bib contains your publications
%\addbibresource{sample.bib}

\begin{document}
\name{李肇中}
\tagline{服务端开发工程师}
%\photo{1.5cm}{Yo_prof.jpg}
\personalinfo{%
  % Not all of these are required!
  % You can add your own with \printinfo{symbol}{detail}
  \email{\url{ldm5235@gmail.com}}
  \phone{+86 18510253518}
  %\mailaddress{Address, Street, 00000 County}
  %\location{Beijing}\\
  \homepage{\url{https://www.lizhaozhong.info}}
  %\homepage{\url{http://lzz5235.github.io}}
  \linkedin{\url{https://www.linkedin.com/in/lizhaozhong/}}
  \github{\url{https://github.com/lzz5235}}
  %% You MUST add the academicons option to \documentclass, then compile with LuaLaTeX or XeLaTeX, if you want to use \orcid or other academicons commands.
%   \orcid{orcid.org/0000-0000-0000-0000}
}

%% Make the header extend all the way to the right, if you want. 
\begin{fullwidth}
\makecvheader
\end{fullwidth}

%% Provide the file name containing the sidebar contents as an optional parameter to \cvsection.
%% You can always just use \marginpar{...} if you do
%% not need to align the top of the contents to any
%% \cvsection title in the "main" bar.
\cvsection[page1sidebar]{项目经历}

\cvevent{摄像头云存储支付系统升级与对接(Java)}{接入小米支付系统,对接摄像头VIP用户,为付费用户提供更多的个性化服务}{2018.10 - 至今}{小米科技}
\begin{itemize}
\setlength{\itemindent}{.2in}
\item 对接小米全球支付(web支付和米家小米钱包SDK支付)
\item 对接apple pay支付
\end{itemize}

\divider

\cvevent{猫眼低功耗设备人脸识别模块研发(Java)}{对接人脸算法模块,将人脸算法集成到米家App,完成猫眼对家人头像\\的管理与识别}{2018.7 - 2018.10}{小米科技}

\begin{itemize}
\item \textbf{设计人脸识别pipeline}
\item \textbf{设计实现人脸识别异步消息系统}
\item \textbf{设计实现人脸识别数据表结构}
\end{itemize}

\divider

\cvevent{摄像头报警推送系统研发(Java)}{对接人形算法模块,将人形算法集成到米家摄像头App,降低报警推送误报率}{2018.3 - 2018.10}{小米科技}

\begin{itemize}
\item \textbf{升级异步消息系统}
\setlength{\itemindent}{.2in}
\item 将内存异步消息队列修改为外部消息队列
\item 增加redis,优化推送逻辑,降低GPU机器load
\setlength{\itemindent}{0in}
\item 对接微信服务号,设计微信报警系统,进行微信报警消息推送
\setlength{\itemindent}{.2in}
\item 设计实现web端播放接口、app端推送开关接口
\setlength{\itemindent}{0in}
\end{itemize}

\divider

\cvevent{船舶轨迹数据存储系统设计与轨迹相似性算法设计实现}{\url{https://github.com/lzz5235/AIS}}{2016.7 - 2018.3}{中国科学院软件研究所}
\begin{itemize}
\item 对多源船舶轨迹数据进行预处理
\item 采用密度聚类算法/层次聚类算法对航迹和停泊点进行挖掘
\item 对聚类结果航迹绘制
\end{itemize}

\divider
%\cvevent{基于QGIS的卫星图像软件的二次开发(C++)}{QGIS软件可以对大量的卫星图像数据进行管理,但是该软件存在大量Bug,\\操作过程中经常崩溃。通过断点跟踪和代码分析进行Bugfix,利用\\GDAL建立图像金字塔优化图片加载缓慢的问题。}{2016.7 - 2018.7}{中国科学院软件研究所}

%\addnextpagesidebar[-1ex]{page2sidebar}

\cvevent{海军作战系统通信协议模块与信息管理模块开发(C++)}{按照接口协议实现发送端和接收端,发送端和接收端采取组播形式通信。}{2016.7 - 2018.7}{中国科学院软件研究所}
\begin{itemize}
\item \textbf{协议为基于UDP的可靠数据传输,完成卫星信息通讯}
\item \textbf{按照需求设计数据库表结构,编写基于Qt的信息管理模块,完成该任务\\规划模块的界面设计和数据库交互}
\end{itemize}

\divider

\cvevent{内存多Bit错误修复内核模块开发(C/Linux Kernel)}{分析当前MCA内核子系统的工作原理,修改Kenrel Machine Check Exception的工作流程,达到服务器Kernal的平稳运行}{2014.8 - 2015.9}{兰州大学}
\begin{itemize}
\item \textbf{使用PALLOC+LXC+CRIU 进行错误抑制与错误隔离,配合使用在线内存检测与离线内存检测方式,\\ 达到服务器的平稳运行}
\item \textbf{编写大量桩模块,进程/内核线程配合 MCA 的处理。}
\end{itemize}

%\cvsection{Projects}

%\cvevent{Project 1}{Funding agency/institution}{}{}
%\begin{itemize}
%\item Details
%\end{itemize}

%\divider

%\cvevent{Project 2}{Funding agency/institution}{Project duration}{}
%A short abstract would also work.

%\medskip

%\cvsection{A Day of My Life}

% Adapted from @Jake's answer from http://tex.stackexchange.com/a/82729/226
% \wheelchart{outer radius}{inner radius}{
% comma-separated list of value/text width/color/detail}
%\wheelchart{1.5cm}{0.5cm}{%
%  6/8em/accent!30/{Sleep,\\beautiful sleep}, 
%  3/8em/accent!40/Hopeful novelist by night,
%  8/8em/accent!60/Daytime job,
%  2/10em/accent/Sports and relaxation,
%  5/6em/accent!20/Spending time with family
%}

%% To start a (new) clear page
%\clearpage

%% If the NEXT page doesn't start with a \cvsection but you'd
%% still like to add a sidebar, then use this command on THIS
%% page to add it. The optional argument lets you pull up the 
%% sidebar a bit so that it looks aligned with the top of the
%% main column.
% \addnextpagesidebar[-1ex]{page3sidebar}

\end{document}
