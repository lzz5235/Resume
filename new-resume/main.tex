%%%%%%%%%%%%%%%%%
% This is Yoann E. A. Le Bihan's CV based on the altacv.cls template
% (v1.1.3, 30 April 2017) written by LianTze Lim (liantze@gmail.com).
% Now compiles with pdfLaTeX, XeLaTeX and LuaLaTeX.
% 
%% It may be distributed and/or modified under the
%% conditions of the LaTeX Project Public License, either version 1.3
%% of this license or (at your option) any later version.
%% The latest version of this license is in
%%    http://www.latex-project.org/lppl.txt
%% and version 1.3 or later is part of all distributions of LaTeX
%% version 2003/12/01 or later.
%%%%%%%%%%%%%%%%

%% If you need to pass whatever options to xcolor
\PassOptionsToPackage{dvipsnames}{xcolor}

%% If you are using \orcid or academicons
%% icons, make sure you have the academicons 
%% option here, and compile with XeLaTeX
%% or LuaLaTeX.
% \documentclass[10pt,a4paper,academicons]{altacv}

%% Use the "normalphoto" option if you want a normal photo instead of cropped to a circle
% \documentclass[10pt,a4paper,normalphoto]{altacv}

\documentclass[10pt,a4paper]{altacv}

%% AltaCV uses the fontawesome and academicon fonts
%% and packages. 
%% See texdoc.net/pkg/fontawecome and http://texdoc.net/pkg/academicons for full list of symbols.
%% 
%% Compile with LuaLaTeX for best results. If you
%% want to use XeLaTeX, you may need to install
%% Academicons.ttf in your operating system's font 
%% folder.


% Change the page layout if you need to
\geometry{left=1cm,right=9cm,marginparwidth=6.8cm,marginparsep=1.2cm,top=1.25cm,bottom=1.25cm,footskip=2\baselineskip}

% Change the font if you want to.

% If using pdflatex:
\usepackage[utf8]{inputenc}
\usepackage[T1]{fontenc}
\usepackage[default]{lato}

% If using xelatex or lualatex:
% \setmainfont{Lato}

% Change the colours if you want to
\definecolor{Mulberry}{HTML}{72243D}
\definecolor{SlateGrey}{HTML}{2E2E2E}
\definecolor{LightGrey}{HTML}{666666}
\colorlet{heading}{Sepia}
\colorlet{accent}{Mulberry}
\colorlet{emphasis}{SlateGrey}
\colorlet{body}{LightGrey}

% Change the bullets for itemize and rating marker
% for \cvskill if you want to
\renewcommand{\itemmarker}{{\small\textbullet}}
\renewcommand{\ratingmarker}{\faCircle}

%% sample.bib contains your publications
%\addbibresource{sample.bib}

\begin{document}
\name{Yoann E. A. Le Bihan}
\tagline{Open Source Expert Consultant (15+ years)}
\photo{2.8cm}{Yo_prof}
\personalinfo{%
  % Not all of these are required!
  % You can add your own with \printinfo{symbol}{detail}
  \email{yoann@yoann.info}
  \phone{+352 691 91 97 98}
  %\mailaddress{Address, Street, 00000 County}
  \twitter{@ylebihan}
  \location{Luxembourg}\\
  \homepage{www.yoann.info}
  \linkedin{linkedin.com/in/yoannlebihan}
  \github{github.com/ylebihan}
  %% You MUST add the academicons option to \documentclass, then compile with LuaLaTeX or XeLaTeX, if you want to use \orcid or other academicons commands.
%   \orcid{orcid.org/0000-0000-0000-0000}
}

%% Make the header extend all the way to the right, if you want. 
\begin{fullwidth}
\makecvheader
\end{fullwidth}

%% Provide the file name containing the sidebar contents as an optional parameter to \cvsection.
%% You can always just use \marginpar{...} if you do
%% not need to align the top of the contents to any
%% \cvsection title in the "main" bar.
\cvsection[page1sidebar]{Experience}

\cvevent{Infrastructure Engineering Transition Manager (contractor)}{Euronext}{January 2017 -- Ongoing}{Paris, France}
\begin{itemize}
\item Ad-interim management of a team of 10 engineers (systems, virtualisation, storage, backup, DBAs)
\item HR management, recruitment (screening and interviews)
\item Interaction and negotiation with suppliers
\item Transition to IaaS/DevOps (industrialisation and Agile provisioning)
\item Multicultural environment (teams in Northern Ireland and Portugal)
\item Highly critical, highly available environment, low latency trading
\item Technical environment: RHEL 5/6/7; Ansible; Satellite 6; HP 3PAR; EMC DataDomain
\end{itemize}

\divider

\cvevent{Senior Linux Expert \& Project Manager (contractor)}{Clearstream Services}{February 2011 -- December 2016}{Luxembourg}
\begin{itemize}
\item \textbf{Red Hat Satellite 6 / Standard Operating Environment}
\setlength{\itemindent}{.2in}
\item Single point of contact for the Clearstream/DB Group
\item Coordination of the SOE project with Red Hat Consulting
\item Project management \& cross-country coordination of resources
\setlength{\itemindent}{.4in}
\item Integration with Jenkins/Git, Puppet, DHCP / DNS
\item Interaction with virtual environments \& cloud project
\setlength{\itemindent}{0in}
\item \textbf{Daily administration}
\setlength{\itemindent}{.2in}
\item RPM packaging, management of internal software builds and middleware packages from third-party vendors (e.g. Red Hat JBoss), Satellite channels management and follow-up
\item L2 \& L3 support / troubleshooting of RHEL and Satellite servers
\item Maintenance of kickstart scripts and deployment documentation
\setlength{\itemindent}{0in}
\item \textbf{Technical point of contact for several projects}
\setlength{\itemindent}{.2in}
\item Docker containerisation PoC
\item Cloud-compliant OS builds for AWS and private cloud solutions
\item Integration of a new subsidiary's DMS into Clearstream's IT
\setlength{\itemindent}{0in}
\item \textbf{DNS migration project}
\setlength{\itemindent}{.2in}
\item Deployment of PowerDNS with replicated MariaDB instances
\item Bind zones automated migration
\setlength{\itemindent}{0in}
\item \textbf{Configuration management \& Compliance checks}
\setlength{\itemindent}{.2in}
\item Puppet PoC design \& deployment
\item Red Hat Satellite 5's configuration channels deployment
\item Development of management scripts based on Satellite's API
\setlength{\itemindent}{0in}
\item \textbf{Servers automated inventory}
\setlength{\itemindent}{.2in}
\item OCSng inventory server deployment \& customisation
\item Massive deployment of OCSng's agent (Linux \& Solaris)
\item Full assets lifecycle workflow design
\end{itemize}

\addnextpagesidebar[-1ex]{page2sidebar}
\clearpage

\cvevent{Linux \& Virtualisation Expert (contractor)}{Ministère de l'intérieur}{May 2010 -- February 2011}{Paris, France}
\begin{itemize}
\item \textbf{Deployment of new Linux hosting environments}
\setlength{\itemindent}{.2in}
\item PXE + kickstart industrial deployment of RHEL 4/5
\item Deployment of hosting platforms based on Debian Linux
\item Deployment of new applications (planning \& resource)
\setlength{\itemindent}{0in}
\item \textbf{Daily management of a thousand Linux servers hosting environment}
\setlength{\itemindent}{.2in}
\item Incident management (in priority for production)
\item Interaction with storage, network, and middleware teams
\setlength{\itemindent}{0in}
\item \textbf{Data center virtualisation project}
\setlength{\itemindent}{.2in}
\item VMware vSphere 4 Enterprise Plus
\item Deployment, configuration and customisation (ESX and vCenter)
\item Best-practices and documentation writing
\item PowerShell scripting (management and administration scripts)
\end{itemize}

\divider

\cvevent{Freelance Consultant \& Entrepreneur}{Self-employed}{June 2002 -- March 2010}{Paris area, France}
Previous experience includes:
\smallskip
\begin{itemize}
\item \textbf{Infrastructure Architect \& Virtualisation Expert at BNP Paribas}\\
		(December 2009 -- March 2010)
\item \textbf{Virtualisation Consultant \& Trainer at Virtao}\\
		(April 2009 -- August 2009)
\item \textbf{Security Consultant at Ingenico}\\
		(December 2008 -- April 2009)
\item \textbf{Linux System Engineer at NYSE Euronext}\\
		(June 2008 -- November 2008)
\item \textbf{Developer \& Linux System Consultant at Meteor Networks}\\
		(January 2008 -- May 2008)
\item \textbf{Executive Director (Founder \& CEO) at Softroad}\\
		(February 2004 -- October 2007)
\item \textbf{Developer \& System Administrator, remotely while a student}\\
		(June 2002 -- January 2004)
\end{itemize}

%\cvsection{Projects}

%\cvevent{Project 1}{Funding agency/institution}{}{}
%\begin{itemize}
%\item Details
%\end{itemize}

%\divider

%\cvevent{Project 2}{Funding agency/institution}{Project duration}{}
%A short abstract would also work.

%\medskip

%\cvsection{A Day of My Life}

% Adapted from @Jake's answer from http://tex.stackexchange.com/a/82729/226
% \wheelchart{outer radius}{inner radius}{
% comma-separated list of value/text width/color/detail}
%\wheelchart{1.5cm}{0.5cm}{%
%  6/8em/accent!30/{Sleep,\\beautiful sleep}, 
%  3/8em/accent!40/Hopeful novelist by night,
%  8/8em/accent!60/Daytime job,
%  2/10em/accent/Sports and relaxation,
%  5/6em/accent!20/Spending time with family
%}

%% To start a (new) clear page
%\clearpage

%% If the NEXT page doesn't start with a \cvsection but you'd
%% still like to add a sidebar, then use this command on THIS
%% page to add it. The optional argument lets you pull up the 
%% sidebar a bit so that it looks aligned with the top of the
%% main column.
% \addnextpagesidebar[-1ex]{page3sidebar}

\end{document}
