\documentclass[11pt,a4paper]{moderncv}

% moderncv themes
%\moderncvtheme[blue]{casual}                 % optional argument are 'blue' (default), 'orange', 'red', 'green', 'grey' and 'roman' (for roman fonts, instead of sans serif fonts)
\moderncvtheme[blue]{classic}                % idem
\usepackage{xunicode, xltxtra}
\XeTeXlinebreaklocale "zh"
\widowpenalty=10000

%\setmainfont[Mapping=tex-text]{文泉驿正黑}

% character encoding
%\usepackage[utf8]{inputenc}                   % replace by the encoding you are using
\usepackage{CJKutf8}

% adjust the page margins
\usepackage[scale=0.8]{geometry}
\recomputelengths                             % required when changes are made to page layout lengths
\setmainfont[Mapping=tex-text]{Hiragino Sans GB}
\setsansfont[Mapping=tex-text]{Hiragino Sans GB}
\CJKtilde

% personal data

%% start of file `template-zh.tex'.
%% Copyright 2006-2012 Xavier Danaux (xdanaux@gmail.com).
%
% This work may be distributed and/or modified under the
% conditions of the LaTeX Project Public License version 1.3c,
% available at http://www.latex-project.org/lppl/.

% 个人信息
\firstname{张}
\familyname{旭婷}
\title{个人简历}                      % 可选项、如不需要可删除本行
\address{190 Lees Ave Unit 1704,ON K1S 5L5}{Ottawa}             % 可选项、如不需要可删除本行
\mobile{+1(613)-265-5462}                         % 可选项、如不需要可删除本行
%\phone{+2~(345)~678~901}                          % 可选项、如不需要可删除本行
%\fax{+3~(456)~789~012}                            % 可选项、如不需要可删除本行
\email{zhangxuting91@yahoo.ca}                    % 可选项、如不需要可删除本行
%\homepage{lizhaozhong.info}                  % 可选项、如不需要可删除本行
%\extrainfo{附加信息 (可选项)}                  % 可选项、如不需要可删除本行
\photo[64pt]{avatar2.jpg}                  % ‘64pt’是图片必须压缩至的高度、‘0.4pt‘是图片边框的宽度 (如不需要可调节至0pt)、’picture‘ 是图片文件的名字;可选项、如不需要可删除本行
%\quote{引言(可选项)}                           % 可选项、如不需要可删除本行

% 显示索引号;仅用于在简历中使用了引言
%\makeatletter
%\renewcommand*{\bibliographyitemlabel}{\@biblabel{\arabic{enumiv}}}
%\makeatother

% 分类索引
%\usepackage{multibib}
%\newcites{book,misc}{{Books},{Others}}
%----------------------------------------------------------------------------------
%            内容
%----------------------------------------------------------------------------------
\begin{document}
\maketitle

\section{教育背景}
\cventry{2016 -- 2013}{本科}{渥太华大学}{Telfer School of Management}{市场营销/人力资源管理}{}
\cventry{2012 -- 2013}{本科}{滑铁卢大学}{Faculty Of Mathematics}{数学专业}{}
\cventry{2010 -- 2012}{高中}{北京中加枫华国际学校}{}{Mathematical/Microeconomics/Statistics}{}

%\section{毕业论文}
%\cvitem{题目}{\emph{题目}}
%\cvitem{导师}{导师}
%\cvitem{说明}{\small 论文简介}

\section{市场 \& 相关经历}
\renewcommand{\baselinestretch}{1.2}
\cventry{2016.8 -- 至今}
{某海军作战系统任务规划模块开发}
{C++,Qt}
{}{项目核心成员}
{该海军作战系统基于已有的内部类库开发,每个功能模块为一个lib文件。本人完成该任务规划模块的界面设计和数据库交互,按照需求设计数据库表结构,配合该作战系统的其他子模块完成既定需求。}

\cventry{2016.7 -- 2016.8}
{某海军后勤管理系统国产化移植和性能优化}
{Java,JavaFx,Oracle}
{}{}
{该作战系统基于 Java Swing 开发,移植系统初期系统运行缓慢。该项目代码行数10W+,通过断点跟踪,分析系统执行路径和性能瓶颈,重写系统中与XML操作相关模块和失效命令,对关键数据结构HashMap进行对象序列化和反序列化,加快启动过程。将该系统数据库从Oracle迁移到国产数据库,添加关键索引以加快数据库存取速率。}

\cventry{2014.8 -- 2015.9}
{内存多Bit错误修复内核模块开发}
{Memory Kernel MCE}
{}{项目核心成员}
{分析当前MCA内核子系统的工作原理,修改Machine Check Exception的工作流程,发信号通知内核线程终止运行,以防止更加严重的panic。对于用户层的进程,使用PALLOC+LXC+CRIU 进行错误抑制与错误隔离。配合使用在线内存检测与离线内存检测方式,达到服务器的平稳运行,而不会因为MCE相关错误而中断服务。编写大量桩模块,进程/内核线程配合 MCA 的处理。}


\section{领导力 \& 校园经历} % (fold)
\cventry{2014-2015}
{国际数学竞赛}{}{}{}{}
\cventry{2013-2014}
{数学建模竞赛}{}{}{}{这个竞争要求选手在36小时内解决一个指定问题,大赛考察学生的数学能力、编程能力和英文写作。}

\section{语言技能}
\cvline{英语}{可用英语流利的交流,拥有出色的英语写作能力。}
\cvline{普通话}{母语}

\section{计算机技能}
\cvline{Office}{Word, Excel, PowerPoint, Access.}
\cvline{文字排版}{\LaTeX}

\closesection{}                   % needed to renewcommands
\renewcommand{\listitemsymbol}{-} % change the symbol for lists

\end{document}


%% 文件结尾 `template-zh.tex'.
